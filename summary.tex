\begin{rubric}{Summary}
\entryy[] Experienced bioinformatics and computational biology professional with a successful track record of combining biology and big data analytics to support organismal enhancement and genomics-enabled research. 
\entryy[] Knowledge and experience with tools and techniques for Next generation sequencing data analysis, including genome assembly and annotation, RNA-Seq, small RNA sequening, Digenome, etc.
\entryy[] Strong foundation in machine learning techniques including Support Vector Machines, deep learning, and Artificial Neural Networks for predictive modeling in bioinformatics.
\entryy[] Ability to develop and implement algorithms for bioinformatics applications using Linux/Unix environments, command-line tools programming languages such as Python, R, and Perl for data analysis and manipulation.
% \entryy[] Hands-on experience in a range of bioinformatics tools and packages such as CLC Genomics, BLAST, FASTA, InterProScan, Psi-Blast, BlastClust, CD-hit, HMM-Pfam, Gene Ontology, etc.{} for sequence alignment, annotation, and analysis.
% \entryy[] Proficient in sequence analysis techniques including homology search, pathway analysis, subcellular localization, PSSM, motif, and domain search for functional annotation and prediction.
% \entryy[] Experience in phylogenetics software such as MEGA, PHYLIP, ClustalW for evolutionary analysis and reconstruction.
% \entryy[] Knowledge of various biological databases including NCBI, UniProt, KEGG, STRING, Pathguide, and IntAct for data retrieval and integration.
% \entryy[] Familiarity with statistical techniques for analysis and interpretation of biological data.
% \entryy[] Experience in using Linux/Unix environments and command-line tools for data analysis and manipulation.
\entryy[] Strong problem-solving and critical thinking skills for complex data analysis and interpretation.
% \text{As an experienced bioinformatics and computational biology professional, I have a proven track record of combining biology and big data analytics to support organismal enhancement and genomics-enabled research. My expertise includes analyzing high throughput data from Next-generation sequencing and developing machine learning-based models for predicting abiotic stress, resistant genes, subcellular localization, and other biological problems. I also excel in developing host-pathogen interaction models and maintaining workflows, tools, web servers, and databases for analysis and data mining.}
\end{rubric}